\documentclass[a4paper,12pt]{article}

\usepackage{fancyhdr}
\usepackage{lastpage}
\usepackage{amsmath}
\usepackage{tikz}
\usepackage{amsfonts}
\usepackage{csvsimple}
\usepackage{graphicx}


\newcommand{\V}[1]{\ensuremath{\vec{#1}}}
\newcommand{\F}[2]{\ensuremath{\frac{#1}{#2}}}
\newcommand{\Q}[1]{\newpage \section*{#1}}
\newcommand{\acc}[1]{\overset{..}{#1}}
\newcommand{\vel}[1]{\overset{.}{#1}}
\newcommand{\prt}[2]{\frac{\partial#1}{\partial#2}}
\newcommand{\LP}{\left(}
\newcommand{\RP}{\right)}

\pagestyle{fancy}
\lhead{Samuel Loomis}
\setlength{\headheight}{15pt}
\chead{Math Methods Homework 1}
\rhead{\thepage\ of \pageref{LastPage}}
\lfoot{}
\cfoot{}
\rfoot{}

\begin{document}
\section*{2. First-Order, Separable Differential Equations}
Find the general solution of each of the followiing differential equations.  Use Boas Chapter 8 Section 2 for review f an apprpriate technique.
\subsection*{(a)}
\begin{align*}
y'&=\F{x^2}{y}\\
\F{dy}{dx}&=\F{x^2}{y}\\
ydy&=x^2dx\\
\int ydy&=\int x^2dx\\
\F{y^2}{2}&=\F{x^3}{3}+C\\
y&=\pm\sqrt{\F{2}{3}x^3+C}
\end{align*}
\subsection*{(b)}
\begin{align*}
xy'-xy&=y\\
x\LP\F{dy}{dx}-y\RP&=y\\
\F{1}{y}\LP\F{dy}{dx}-y\RP&=\F{1}{x}\\
\F{dy}{y}-dx&=\F{dx}{x}\\
\F{dy}{y}&=\LP\F{1}{x}+1\RP dx\\
\int\F{dy}{y}&=\int\LP\F{1}{x}+1\RP dx\\
ln(y)&=ln(x)+ln(c)+x\\
y&=cx+e^x
\end{align*}
\subsection*{(c)}
\begin{align*}
xy'-xy&=x\\
\F{dy}{dx}-y&=1\\
\F{dy}{dx}&=y+1\\
\F{dy}{y+1}&=dx\\
ln(y+1)&=x+c\\
y+1&=e^{x+c}\\
y&=e^xe^c-1\\
y&=ce^x-1
\end{align*}
\section*{3. Quadratic Drag}
An object is moving in one direction which is only subject to one force, quadratic drag, given by:
\begin{align*}
-cv^2
\end{align*}
where $v$ is the velocity of the object and $c$ is a constant. Write the equation of motion for this object and solve for the velocity at any time $t$ for an initial velocity of $v_0$.\\
\\
Assuming $c$ already has the mass of the object divided out, the acceleration of the object is $a=-cv^2$.\\
\\
$a$ is the time derivative of the velocity $v$ and $v$ is the time derivative of the position $x$ thus:
\begin{align*}
v&=\F{dx}{dt}\\
a&=\F{dv}{dt}=\F{d^2x}{dt^2}
\end{align*}
Solving for $v$:
\begin{align*}
\F{dv}{dt}&=-cv^2\\
-\F{dv}{v^2}&=cdt\\
\F{1}{v}&=ct+C\\
v&=\F{1}{ct+C}
\end{align*}
Since $v(t=o)=v_0$, $\F{1}{C}=v_0$
\begin{align*}
v&= \F{v_0}{cv_0t+1}
\end{align*}
The above shows the equation for $v(t)$, seperating variables one more time will give the equation of motion.
\begin{align*}
\F{dx}{dt}&=\F{v_0}{cv_0t+1}\\
\int dx&=\int\F{v_0dt}{cv_0t+1}\\
x&=\F{1}{c}ln(cv_0t+1)+C
\end{align*}
\end{document}