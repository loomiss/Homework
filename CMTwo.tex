\documentclass[a4paper,12pt]{article}

\usepackage{fancyhdr}
\usepackage{lastpage}
\usepackage{amsmath}
\usepackage{tikz}
\usepackage{amsfonts}
\pagestyle{fancy}
\lhead{Samuel Loomis}
\setlength{\headheight}{15pt}
\chead{Classical Mechanics HW 1}
\rhead{\thepage\ of \pageref{LastPage}}
\lfoot{}
\cfoot{}
\rfoot{}

\begin{document}

\section*{Question 1}
Assumption: v is a velocity, and will be negative if moving down.

(a): \[m\frac{dv}{dt}=-mg+c_1v\] If the ball moves up, $c_1v$ would be
positive and thus accelerate, which is not a drag force.  If the ball
moves down, $c_1v$ would be negative, and again, accelerate.  This
situation will not work for either case.

(b):\[m\frac{dv}{dt}=-mg-c_1v\] If the ball moves up, $-c_1v$ would be
negative, which would accelerate down and counter the motion of the
ball.  If the ball moves down, $-c_1v$ would be positive, and again,
counter the motion of the ball.  This situation will work for either case.

(c):\[m\frac{dv}{dt}=-mg+c_2v^2\] Whether the ball moves up or down,
$c_2v^2$ will be positive, and thus accelerate the ball if its moving
up or counter the motion if its moving down.  This situation will only
work if the ball is moving down.

(d):\[m\frac{dv}{dt}=-mg-c_2v^2\] Whether the ball moves up or down,
$-c_2v^2$ will be negative, and thus accelerate the ball if its moving
down or counter the motion if its moving up.  This situation will only
work if the ball is moving up.

\section*{Question 2}

Terminal velocity $V_{term}$ happens when the sum of the forces equal
0 and thus, the falling object has no acceleration.
\[m\frac{dv}{dt}=0\]
A falling object in air has forces approximately equal to gravity and
a linear drag + a quadratic drag.
\[m\frac{dv}{dt}=-mg+c_1v+c_2v^2\]
The drag forces are always counter to the movement.  Setting the
equation to 0 and solving for v will give and expression for
$v_{term}$.
\[0=-mg+c_1v+c_2v^2\]
(a) If the air resistance is only quadratic, the expression becomes:
\[0=-mg+c_2v^2\]Solving for v gives\[v_{term}=\sqrt{\frac{mg}{c_2}}\]
(b) If the air resistance has both linear and quadratic terms, the
expression can be solved using the quadratic formula:
\[v_{term}=\frac{-c_1\pm\sqrt{c_1^2+4c_2mg}}{2c_2}\]

\section*{Question 3}
\[\left|\frac{c_2v^2}{c_1v}\right|=\frac{0.22v\left|v\right|D^2}{(1.55\ast10^{-4})vD}=(1.4\ast10^3)\left|v\right|D\](a):
I
will assume that the quadratic term dominates over the linear term
when the ratio is 100 or higher.  Solving for v:
\[100=(1.4\ast10^3)\left|v\right|D\]
\[\left|v\right|=\frac{100}{(1.4\ast10^3)D}\approx7.14\frac{m}{s}\]
Speeds below 7.14 $\frac{m}{s}$ will be dominated by the linear drag
and speeds above will be dominated by the quadratic drag.\\
(b): From
Q.2. $v_{term}=\frac{-c_1\pm\sqrt{c_1^2+4c_2mg}}{2c_2}$\ substituting
in the values
$c_1=(1.55\ast10^{-4})D$,$c_2=0.22D^2$,$D=0.01$,$m=0.2$\ and $g=9.8$ the
equation becomes:
\begin{align*}
v_{term}&=\frac{-(1.55\ast10^{-4})\ast0.01\pm\sqrt{((1.55\ast10^{-4})\ast0.01)^2+4\ast0.22\ast0.01^2\ast0.2\ast9.8}}{2\ast0.22\ast0.01^2}\\
v_{term}&=\frac{-1.55*10^{-6}\pm\sqrt{1.7248*10^{-4}}}{4.4*10^{-5}}\\
v_{term}\approx\pm298\frac{m}{s}
\end{align*}
Since negative speed for $v_{term}$ is kinda useless:
$v_{term}=298\frac{m}{s}$

(c): Seeing as how speeds above 7.14$\frac{m}{s}$\ were considered to
be dominated by the quadratic drag term, if I had to choose which one
to use, and only use, it would be the quadratic term for drag.  The
speeds are deep in the quadratic realm.

\section*{Question 4}
(a): I am going to again assume the quadratic term dominates at a
ratio of 100 or higher, and solve for v:
\[\left|v\right| =
\frac{100}{(1.4\ast10^3)\ast(10^{-6})}\approx7.14*10^4\frac{m}{s}\]
According to this calculation, to be dominated by the quadratic drag,
the oil drop will have to be traveling in excess of
71.4$\frac{km}{s}$.  Anything below is dominated by the linear drag
force.

(b): At this point, I am taking the equation in Q.3. and substituting
$c_1=(1.55\ast10^{-4})D$,$c_2=0.22D^2$,$D=0.000001$,$m=10^{-15}$\ and
$g=9.8$.  Using these values, $v_{term}$ is calcualted to be about
$6.3*10^{-5}\frac{m}{s}$.

(c):  $v_{term}$ for the oil drop is well within the range of speeds
to be considered linear drag dominant.  Linear drag is the term I
would focus on for this experiment.

\section*{Question 5}
The velocity of the cyclist coasting to a stop will include the
$f_{fr}$ and $f_{air}$.
\begin{align*}
m\frac{dv}{dt}&=f_{air}+f_{fr}\\
&=-0.20\frac{N}{(\frac{m}{s})^2}v^2-3N
\end{align*}
Separation of variables leads to:
\begin{align*}
\frac{mdv}{-0.20\frac{N}{(\frac{m}{s})^2}v^2-3N}=dt
\end{align*}
Integrating both sides:
\begin{align*}
\int\frac{mdv}{-0.20\frac{N}{(\frac{m}{s})^2}v^2-3N}&=\int dt\\
(from\  Wolfram\ Alpha)\frac{-1.29s\ m\  tan^{-1}(\frac{0.258s\ v}{m})}{kg}\approx
  t
\end{align*}
Evaluating at $v=20,15,10,5,0\frac{m}{s}$ and $m=80kg$gives values for t:
\begin{align*}
t(20)=-142.4s,\  t(15)=-136.0s,\  t(10)=-123.9s,\  t(5)=-94.1s,\  t(0)=0s
\end{align*}
To go from $20\frac{m}{s}$\ to $15\frac{m}{s}$ will take about 6.4s.\\
To go from $10\frac{m}{s}$\ to $5\frac{m}{s}$ will take about 29.8s.\\
To go from $20\frac{m}{s}$\ to a full stop will take about 142.4s.
\section*{Question 6}
The equation of motion for this case would be:
\begin{align*}
m\frac{dv}{dt}=-mg+cv^2
\end{align*}
From before $v_{term}=\sqrt{\frac{mg}{c}}$, solving for c and
substituting into the equation for motion:
\begin{align*}
c&=\frac{mg}{v_{term}^2}\\
\frac{dv}{dt}&=g\left(1-\frac{v^2}{v_{term}^2}\right)
\end{align*}
Separating and check here for proper integration limits (fix me) integrating:
\begin{align*}
\frac{dv}{1-\frac{v^2}{v_{term}^2}}&=gdt\\
\frac{v_{term}}{g}arctanh\left(\frac{v}{v_{term}}\right)&=t
\end{align*}
Taking the tanh of both sides and solving for v gives:
\begin{align*}
arctanh\left(\frac{v}{v_{term}}\right)&=\frac{gt}{v_{term}}\\
\frac{v}{v_{term}}&=tanh\left(\frac{gt}{v_{term}}\right)\\
v&=v_{term}tanh\left(\frac{gt}{v_{term}}\right)
\end{align*}
Integrating to find the position:
\begin{align*}
y=\frac{v_{term}^2}{g}ln\left[cosh\left(\frac{gt}{v_{term}}\right)\right]
\end{align*}
\end{document}


