\documentclass[a4paper,12pt]{article}

\usepackage{fancyhdr}
\usepackage{lastpage}
\usepackage{amsmath}
\usepackage{tikz}
\usepackage{amsfonts}
\usepackage{csvsimple}
\usepackage{graphicx}

\pagestyle{fancy}
\lhead{Samuel Loomis}
\setlength{\headheight}{15pt}
\chead{Table of words from the first reading.}
\rhead{\thepage\ of \pageref{LastPage}}
\lfoot{}
\cfoot{}
\rfoot{}

\begin{document}
\begin{center}
\bf{\huge Samuel Loomis}\\
\bf{\large English 106}\\
\bf{\large First Reading Table of Words}\\
\bf{\large 4/8/2014}
\end{center}
\newpage
\section*{Infant Sorrow}
\begin{center}
	\begin{tabular}{|l|l|l|p{2.5in}|}
	\hline
	Line \#& Word 1 & Word 2 & Reasoning \\ \hline
	1&groaned&wept&The infant is picking up on the pain and sorrow of their parents.\\ \hline
	2&dangerous&lept&The infant is leaping into a world of danger.  What awaits?\\ \hline
	3&helpless&naked&To be honest, thats half the words in this line, however, being helpless and naked provides a sense of extreme peril\\ \hline
	4&fiend&hid&A hidden fiend?  What were the infants experiences before emerging?  Was there a conscious that was forgotten at birth?\\ \hline
	5&struffling&father's&Struggling adds to the feeling of peril and helplessness, and father's hands are extremely foreign, not the same touch or feel as the mother who carried the infant.\\ \hline
	6&striving&bands&The infant is fighting to be free and yet is bound and unable to act on their own.  More helpless sorrow feeling.\\ \hline
	7&bound&weary&More helpless feeling.\\ \hline
	8&sulk&mother's&The infant is finally accepting the unavoidable.  With no power, they are left to inaction.\\ \hline
	\end{tabular}
\end{center} 
\section*{The Plain Sense of Things}
\begin{center}
	\begin{tabular}{|l|l|l|p{2.5in}|}
	\hline
	Line \#& Word 1 & Word 2 & Reasoning \\ \hline
	1&fallen&leaves&Fallen leaves can stand for the end of life, or an age gone by.  The reference sets up the poem as a feeling of loss and regret.\\ \hline
	2&plain&sense&Plain sense speaks of perfect hind sight, what did all our actions actually mean?\\ \hline
	\end{tabular}
\end{center}
\begin{center}
	\begin{tabular}{|l|l|l|p{2.5in}|}
	\hline
	Line \#& Word 1 & Word 2 & Reasoning \\ \hline
	3&end&imagination&The end of imagination seems to me as if its a great travesty.  To loose the ower of imagination is to almost fall in to a deep pit of despar.\\ \hline
	4&inert&savior&Inert savior leads one to think back on the hope of ones life, "everything for a reason", and feel utterly lost and betrayed.\\ \hline
	5&difficult&choose&How does one accurately describe the loss of hope?  Is there an adjective that can express the utter despair?\\ \hline
	6&cold&sadness&Cold, helpless, sadness, despair.\\ \hline
	7&great&minor&The idea of something so grand dwindling into minor nothingness...  This guy was having a bad day.\\ \hline
	8&turban&lessened&Same idea here, turbans was the greatness of royalty, dwindled into the lesser state of things.\\ \hline
	9&badly&needed&More decay, more rundown images.\\ \hline
	10&old&slants&Old decaying and falling apart.  The image Stevens has is dim and almost desolate.\\ \hline
	11&effort&failed&Really?  Has life been so bad?  Almost every line is saying the same thing, despair and desolation.\\ \hline
	12&men&flies&Mens actions are as insignificant as flies.\\ \hline
	13&absence&imagination&The absence of imagination is a dark thought.\\ \hline
	14&imagined&great&The pond being great was imagined...  Was it truly great?\\ \hline
	15&sense&without&The sense of things looking back, is without granduer, without the reflections, the pond seems lifeless.\\ \hline
	16&dirty&silence&Dirty silence just paints more of a descolate picture.  Stevens was in a very dark place when he wrote this.\\ \hline
	17&rat&see&What a rat sees, squalor and grime.\\ \hline
	18&great&waste&Stevens uses grandor decaying into waste alot.  A supposedly grand life that in reality was an illusion coating over the squalor of reality.\\ \hline
	\end{tabular}
\end{center}
\begin{center}
	\begin{tabular}{|l|l|l|p{2.5in}|}
	\hline
	Line \#& Word 1 & Word 2 & Reasoning \\ \hline
	19&imagined&inevitable&The grand life was imagined, and inevitably it was revealed for the farse it actually was.\\ \hline
	20&necessity&requires&The view Stevens wrote about was a necessity to accurately capture the feeling of loss one likely experiences in the late stages of life\\ \hline
	\end{tabular}
\end{center} 
\section*{Those Winter Sundays}
\begin{center}
	\begin{tabular}{|l|l|l|p{2.5in}|}
	\hline
	Line \#& Word 1 & Word 2 & Reasoning \\ \hline
	1&too&early&On Sundays, Hayden's father would wake up early, before the rest of the family.  "Too" means that this was something he did every day, not just on Sunday.\\ \hline
	2&blueblack&cold&Hayden's father was getting up in the uncofortable cold, so that he could make the house warm for his family.\\ \hline
	3&cracked&ached&His hands were cracked and ached from the week of hard labor, and yet he still thought of his family before his own comfort.\\ \hline
	4&labor&weather&His hands were cracked and painfull from the hard labor in the elements.  The life he had set before him, likely was not chosen, however, he braved the elements and endured the labor to provide.\\ \hline
	5&no&thanked&Hayden't father did this out of his sense of duty and protection for his family, no thanks were needed, and none were given.\\ \hline
	6&wake&breaking&Hayden would wake in the morning and hear his father preparing the firewood, however, he did not get up to help.\\ \hline
	\end{tabular}
\end{center}
\begin{center}
	\begin{tabular}{|l|l|l|p{2.5in}|}
	\hline
	Line \#& Word 1 & Word 2 & Reasoning \\ \hline
	7&warm&call&Hayden's father would wait till the house was warmed before getting people out of bed.  He suffered through to provide comfort for his family.\\ \hline
	8&slowly&rise&Even though another had prepared a warm house for him, Hayden would still rise slowly, likely begrudging being woken at all.  Completely overlooking the kindness performed for him.\\ \hline
	9&chronic&angers&The house Hayden grew up in seems to have been a disfunctional one, much like most families in the world.  However, his father still provided a warm home for them to wake up in and worked to provide the shelter and food.\\ \hline
	10&speaking&indifferently&The indifferent speach shows the lack of thought Hayden had in regards to his father's kindness.  He just took it for granted.\\ \hline
	11&driven&out&Hayden's father provided the warm house for Hayden to wake into.\\ \hline
	12&as&well&On top of everything else, Hayden's father even polished Hayden's shoes for church.  The thoughtfulness of his father was extensive.\\ \hline
	13&what&know&Looking back on his childhood, Hayden is just now realizing what his father did for him.  I imagine, he wishes to go back and offer his thanks, and appreciation.\\ \hline
	14&lonely&offices&Love's lonely offices?  Hayden's father loved Hayden, and likely didn't know how to express it but through his morning ritual and daily work.  To be unable to see this as a child would make the home feel very much like a cold workplace.\\ \hline
	\end{tabular}
\end{center}
\section*{Daddy}
This section I am going to choose two words per stanza.  I am assuming you don't want a 10 page report on the poem "Daddy."
\begin{center}
	\begin{tabular}{|l|l|l|p{2.5in}|}
	\hline
	Stanza \#& Word 1 & Word 2 & Reasoning \\ \hline
	1&shoe&lived&Plath's father is refered to as a "black shoe" in which she lived as a foot.  He was her shield to the wold she traveled in.\\ \hline
	2&kill&before&Plath had some animosity to exress t'words her father, but he died before she could.\\ \hline
	3&pray&recover&Plath misses her father, and prays for him to return to her.\\ \hline
	4&flat&roller&War has flattened many people as if they were a giant steamroller and people had nowhere to run.\\ \hline
	5&root&talk&The town where her father grew up was a common named town, and she couldn't find his family or even the experiences that made him the man he was, thus she had no way to talk to her lost father.\\ \hline
	6&obscene&language&The German language reminded Plath of her father, and when she heard it, lots of regret would flood back to her, thus making the language an obscenity in her ears.\\ \hline
	7&talk&think&Plath's loss of her father disoriented her and made her loose her heritage as well.  Talking and thinking like a Jew at the time of World War II in Germany is to be among the most socially outcast set of people.  She lost herself and felt helpless, like many of the Jews did in that time.\\ \hline
	8&pure&true&Plath's motherly heritage was not pure or true.\\ \hline
	\end{tabular}
\end{center}
\begin{center}
	\begin{tabular}{|l|l|l|p{2.5in}|}
	\hline
	Stanza \#& Word 1 & Word 2 & Reasoning \\ \hline	9&always&scared&Plath has always been scared of the imposing figure of her father. The "pure" German, imposing and grand of stature.  She was of mixed stock, and thus had no place in the new Germany.\\ \hline
	10&boot&brute&The Nazi party of Germany was brutal.\\ \hline
	11&cleft&devil&Plath's father likely had a prominent chin with cleft indention, the devil would have cleft feat, however, she still felt as though that was what he was.  There is much adoration and fear of her father.\\ \hline
	12&ten&twenty&Plath's father died when she was 10, and she missed him so much that she tried to commit suicide at the age of 20.\\ \hline
	13&together&model&Plath was rescued and recovered, and she made a model of of her father to allow her to cope.\\ \hline
	14&rack&do&The model Plath made was the man she married.  She found someone just like him, and even though it was a torture, she married him.\\ \hline
	15&vampire&drank&The mand Plath married sucked the life out of her.\\ \hline
	16&dancing&stamping&There is much rejoicing that she has broken free from her husband, and no longer under the sway of her father's influence.\\ \hline
	\end{tabular}
\end{center}  
\end{document}


