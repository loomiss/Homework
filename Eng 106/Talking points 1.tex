\documentclass[a4paper,12pt]{article}

\usepackage{fancyhdr}
\usepackage{lastpage}
\usepackage{amsmath}
\usepackage{tikz}
\usepackage{amsfonts}
\usepackage{csvsimple}
\usepackage{graphicx}

\pagestyle{fancy}
\lhead{Samuel Loomis}
\setlength{\headheight}{15pt}
\chead{English Talking Points 1}
\rhead{\thepage\ of \pageref{LastPage}}
\lfoot{}
\cfoot{}
\rfoot{}

\begin{document}
\subsection*{Infant Sorrow}
My first impression of the title "Infant Sorrow" is one of a bit of confusion.  What does an infant have to be sorrowful for?  Isn't sorrow a learned response to pain or loss as we develop?

After reading the poem... I can see a bit more what is meant by "Infant Sorrow."  Loosing the warmth and comfort of their mother's womb and stepping out into a new world, with no control or ability to survive on ones own.  Putting our experiences in the world into a newborn's situation can make it seem very sorrowful.

\subsection*{Daddy}
The title "Daddy" doesn't offer a lot of context in which to develop an impression.  Daddy is a word I can't remember ever using, and as such, has no real meaning for me.  As long as I can remember I have called my father by dad or pop.  My impression of the title "Daddy" is not really substantial.

After reading this poem... I get the feeling that "Daddy" was a word spoken with longing at times, and at other times, spat out like a vile curse.  Much animosity is detected, however, she still wishes to know her daddy and misses him.

\subsection*{Those Winter Sundays}
Those winter Sundays makes me remember the warm fire in the living room after church.  Curled up on the couch watching Disney movies with the family.  Many good thoughts.

After reading the poem, Those winter Sundays are a cord of regret.  The kid, Hayden?, remembers how thoughtless he was twords the generosity of his father.

\subsection*{The Plain Sense of Things}
The plain sense of things makes me think of common sense.  A guide I try to follow and that I find lacking in many people these days.

After reading the poem, the plain sense of things is the way we look back on our life for what it was, stripped of the granduer we saw as we pursued our goals and desires.  

\end{document}


