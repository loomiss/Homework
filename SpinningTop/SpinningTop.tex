\documentclass[a4paper,12pt]{article}

\usepackage{fancyhdr}
\usepackage{lastpage}
\usepackage{amsmath}
\usepackage{tikz}
\usepackage{amsfonts}
\pagestyle{fancy}
\lhead{Samuel Loomis}
\setlength{\headheight}{15pt}
\chead{Precession and Nutation of a Spinning Top}
\rhead{\thepage\ of \pageref{LastPage}}
\lfoot{}
\cfoot{}
\rfoot{}

\begin{document}
The precession and nutation of a spinning top is an analysis of a rotating rigid body system.


\subsection*{Inertia Tensor}


Products of Inertia:
\begin{align*}
L_x=I_{xz}\omega&\ \ and\ \ L_y=I_{yz}\omega\\
where:\ \ &I_{xz}=-\sum m_\alpha x_\alpha z_\alpha,\\
&I_{yz}=-\sum m_\alpha y_\alpha z_\alpha,\\
&I_{zz}=\sum m_\alpha(x_\alpha^2+y_\alpha^2)
\end{align*}
Inertia Tensor:
\begin{align*}
\left( \begin{array}{ccc}
I_{xx} & I_{xy} & I_{xz}\\
I_{yx} & I_{yy} & I_{yz}\\
I_{zx} & I_{zy} & I_{zz} \end{array} \right)
\end{align*}
The spining top has axial symetry along its core, (upright $z$-axis), this property makes the off diagonal components of the inertia tensor all equal to 0.
The diagonals:
\\
For this example, the top will be portrayed as a uniform density cone.
\\
The $I_{zz}$ product of inertia using cylindrical coordinates is:
\begin{align*}
\varrho&=M/V=\frac{3M}{\pi R^2h}\\
r&=\frac{Rz}{h}
\end{align*}
\begin{align*}
I_{zz}=\varrho\int_{V}dV\rho^2&=\varrho\int_0^hdz\int_0^{2\pi}d\phi\int_0^r\rho d\rho\rho^2\\
&=
\end{align*}
\end{document}


