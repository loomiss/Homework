\documentclass[a4paper,12pt]{article}

\usepackage{fancyhdr}
\usepackage{lastpage}
\usepackage{amsmath}
\usepackage{tikz}
\usepackage{amsfonts}
\pagestyle{fancy}
\lhead{Samuel Loomis}
\setlength{\headheight}{15pt}
\chead{Classical Mechanics HW 1}
\rhead{\thepage\ of \pageref{LastPage}}
\lfoot{}
\cfoot{}
\rfoot{}

\begin{document}

\section*{Question 1}
Assumption: v is a speed, and will not have a negative sign attached
to it, no matter which direction it is going.

(a): \[m\frac{dv}{dt}=-mg+c_1v\] Could only represent a situation
where the ball moves down.  The drag force described by $c_1$ is
positive and thus must point in the upward direction.  If v moved up,
it would be accelerating v and thus v can only be moving down in this
case.

(b):\[m\frac{dv}{dt}=-mg-c_1v\] Represents a situation where the ball
moves up.  If the ball moves down, the force would accelerate v and
thus v can only be moving up in this case.

(c):\[m\frac{dv}{dt}=-mg+c_2v^2\] This case cannot be represented by
the ball moving up or down.  Since v is squared, no matter if the v is
positive or negative, the overall effect would be an acceleration of
v.

(d):\[m\frac{dv}{dt}=-mg-c_2v^2\] This case represents a ball that can
be moving either way.  Again since vi is squared, the direction
doesn't change teh equation, and since $c_2$ is negative, the ball
will always experience a negative drag force.

\section*{Question 2}

\section*{Question 3}
\[\left|\frac{c_2v^2}{c_1v}\right|=\frac{0.22v\left|v\right|D^2}{(1.55\ast10^{-4})vD}=(1.4\ast10^3)\left|v\right|D\](a):
I
will assume that the quadratic term dominates over the linear term
when the ratio is 100 or higher.  Solving for v:
\[100=(1.4\ast10^3)\left|v\right|D\]
\[\left|v\right|=\frac{100}{(1.4\ast10^3)D}\approx7.14\frac{m}{s}\]
Speeds below 7.14 $\frac{m}{s}$ will be dominated by the linear drag
and speeds above will be dominated by the quadratic drag.\\
(b): The velocity of the ball \[m\frac{dv}{dt}=-mg+c_1vD+c_2v^2D^2\]
\end{document}


