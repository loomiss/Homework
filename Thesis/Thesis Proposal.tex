\documentclass[a4paper,12pt]{article}

\usepackage{fancyhdr}
\usepackage{lastpage}
\usepackage{amsmath}
\usepackage{tikz}
\usepackage{amsfonts}
\usepackage{csvsimple}
\usepackage{graphicx}
\pagestyle{fancy}

\lhead{Samuel Loomis}
\setlength{\headheight}{15pt}
\chead{Thesis Proposal}
\rhead{\thepage\ of \pageref{LastPage}}
\lfoot{}
\cfoot{}
\rfoot{}

\begin{document}
\section*{Proposal Summary}
(To be written after the paper)

\section*{Project Description}

\subsection*{Why is it important?}
Finding better ways to simulate simple homogeneous and inhomogeneous
fluids will allow us to study the fluids and predict how they may act
without spending vast resources on physical experimental setups. For
example, scientists have the capability of taking fluids down to very
cold temperatures, but it takes large amounts of energy to do so. If
we find a reliable simultaion that can be used instead, computers can
be tasked at running the simulations in many configurations. This will
save the energy required to physically test the configuration.

\subsection*{Why Soft Spheres?}
Currently hard sphere simulations are widely used (Verify) and can
give decently accurate results, but are still not as accurate as we
like. (melting point x2?-Verify and add) Hard sphere simulations are
limited in capabilities. Hard spheres have either 0 or infinite
potential interactions, they are either touching or not and are not
allowed to overlap. One way to attempt to improve the model is to
allow the spheres to overlap or soften their edges. By taking the
basis of the hard sphere model, and applying a potential interaction
to the spheres we hope to find a better more accurate model for
fluids.

\subsection*{Current Focus}
A Monte-Carlo program is being modified to simulate a soft sphere
theory. The data will be used to verify a density functional theory
for soft spheres. (Density functional theory-What is it? Add info)

\subsection*{Future Plans}
The Monte-Carlo simulation could be used to test fluid transitional
characteristics. By holding the temperature fixed and plotting
pressure as a function of the number of spheres used, the phase
transition of the simple fluid could be analyzed. The plot should show
a line of pressure of which the number of sphere’s below it will be in
a loose flowing state, and the shperes that show pressure data above
will be in a more crystalline shape.

\subsection*{Timeline}

\subsubsection*{Ongoing Tasks}
- Debug code.\\
- Create code to plot and analyze data.

\subsubsection*{July 2013}
- Run homogeneous soft sphere Monte-Carlo for low densities in order to
calculate g(r) and compare with analytic low density equation.\\
- Run homogeneous soft sphere Monte-Carlo to obtain pressure data at
various temperatures and densities.

\subsubsection*{August 2013}
- Debug code.\\
- Compare pressure data to vivial expansion to verify correct theory.

\subsubsection*{September 2013}
- Run inhomogeneous soft sphere Monte-Carlo near hard walls with
different packing fractions and temperatures to find density
equatioins n(z).

\subsubsection*{October 2013}
- Run more inhomogeneous soft sphere Monte-Carlo near hard walls to
figure out the number of spheres needed to obtain round number
densities.

\subsubsection*{November 2013}
- Run inhomogeneous soft sphere Monte-Carlo with a spherical inner
wall.\\
- Compare Monte-Carlo resutls with the DFT results.

\subsubsection*{December 2013}
- Alter soft sphere Monte-Carlo to handle soft walls and/or walls with
attraction.

\subsubsection*{January 2014}
- Begin studying DFT side of research.\\
- Run DFT code.

\section*{References Cited}

(To be added later.)

\subsection*{Interesting DFT Resource, may or may not use}

http://www.uam.es/personal\_pdi/ciencias/jcuevas/Talks/JC-Cuevas-DFT.pdf

\section*{Facilities, Equipment and Other Resources}
The similation programs and data are stored and executed from a secure
redundant backup system.


\end{document}
