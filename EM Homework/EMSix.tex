\documentclass[a4paper,12pt]{article}

\usepackage{fancyhdr}
\usepackage{lastpage}
\usepackage{amsmath}
\usepackage{tikz}
\usepackage{amsfonts}
\usepackage{graphicx}

\newcommand{\V}[1]{\ensuremath{\vec{#1}}}
\newcommand{\F}[2]{\ensuremath{\frac{#1}{#2}}}
\newcommand{\Q}[1]{\newpage \section*{#1}}
\newcommand{\acc}[1]{\overset{..}{#1}}
\newcommand{\vel}[1]{\overset{.}{#1}}
\newcommand{\prt}[2]{\frac{\partial{#1}}{\partial{#2}}}


\pagestyle{fancy}
\lhead{Samuel Loomis}
\setlength{\headheight}{15pt}
\chead{Electrodynamics 6}
\rhead{20 November 2013}
\lfoot{}
\cfoot{\thepage\ of \pageref{LastPage}}
\rfoot{}

\begin{document}

\section*{Problem 1}
Find the monochromatic plane wave solution in Lorentz gauge, and derive the energy flux density.\\
\\
The equations for Electric and Magnetic field 
Using Maxwell's equations:
\begin{align*}
(i)&~~\V{\nabla}\cdot\V{E}=\F{1}{\epsilon_0}\rho,  &  (iii)&~~\V{\nabla}\times\V{E}=-\prt{\V{B}}{t},\\
(ii)&~~\V{\nabla}\cdot\V{B}=0,&  (iv)&~~\V{\nabla}\times\V{B}=\mu_0\V{J}+\mu_0\epsilon_0\prt{\V{E}}{t}.
\end{align*}
In statics, the curl of $\V{E}=0$ however in dynamics, the curl of $\V{E}\ne0$ and the divergence of $\V{B}$ is still 0. Rather than using the electric potential, the magnetic potential will be used:
\[\V{B}=\V{\nabla}\times\V{A}\]
The curl of $\V{E}$ becomes:
\begin{align*}
\V{\nabla}\times\V{E}&=-\prt{\V{B}}{t}=-\prt{}{t}(\V{\nabla}\times\V{A})\\
\V{\nabla}\times\V{E}&+\V{\nabla}\times\prt{\V{A}}{t}=\V{\nabla}\times\left(\V{E}+\prt{\V{A}}{t}\right)=0
\end{align*}
Since the curl of this thing is equal to 0, it can be set as the gradient of a scalar:
\begin{align*}
\V{E}+\prt{\V{A}}{t}&=-\V{\nabla}V\\
\V{E}&=-\V{\nabla}V-\prt{\V{A}}{t}
\end{align*}
Maxwell's (i) becomes:
\[\nabla^2V+\prt{}{t}(\V{\nabla}\cdot\V{A})=-\F{1}{\epsilon_0}\rho\]
Maxwell's (iv) becomes:
\[\V{\nabla}\times(\V{\nabla}\times\V{A})=\mu_0\V{J}-\mu_0\epsilon_0\V{\nabla}\left(\prt{V}{t}\right)-\mu_0\epsilon_0\prt{^2\V{A}}{t^2}\]
The curl of the curl of the magnetic potential is:
\[\V{\nabla}\times(\V{\nabla}\times\V{A})=\V{\nabla}(\V{\nabla}\cdot\V{A})-\nabla^2\V{A}\]
\[\V{\nabla}(\V{\nabla}\cdot\V{A})-\nabla^2\V{A}=\mu_0\V{J}-\mu_0\epsilon_0\V{\nabla}\left(\prt{V}{t}\right)-\mu_0\epsilon_0\prt{^2\V{A}}{t^2}\]
Rearranging:
\[\left(\nabla^2\V{A}-\mu_0\epsilon_0\prt{^2\V{A}}{t^2}\right)-\V{\nabla}\left(\V{\nabla}\cdot\V{A}+\mu_0\epsilon_0\prt{V}{t}\right)=-\mu_0\V{J}\]
In the Lorenz gauge:
\[\V{\nabla}\cdot\V{A}=-\mu_0\epsilon_0\prt{V}{t}\]
The above becomes:
\[\left(\nabla^2\V{A}-\mu_0\epsilon_0\prt{^2\V{A}}{t^2}\right)=-\mu_0\V{J}\]
No free charge:
\[\nabla^2V=\mu_0\epsilon_0\prt{^2V}{t^2}\]
No free current:
\[\nabla^2\V{A}=\mu_0\epsilon_0\prt{^2\V{A}}{t^2}\]


\Q{Problem 2}
In the monochromatic plan wave solution using Coulomb gauge, we have $\V{A}=\V{A}_0exp[i(\V{k}\cdot\V{r}-\omega t)]$.  In class, we only considered the case of a linear polarized light, meaning $\V{A}_0$ is a real vector.  Now when $\V{A}_0=A_x\hat{x}+iA_y\hat{y}$, derive the corresponding (real) $\V{E}$ and $\V{B}$ field, as well as the energy density and energy flux density.
\end{document}


