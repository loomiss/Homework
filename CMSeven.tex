\documentclass[a4paper,12pt]{article}

\usepackage{fancyhdr}
\usepackage{lastpage}
\usepackage{amsmath}
\usepackage{tikz}
\usepackage{amsfonts}
\usepackage{graphicx}

\newcommand{\V}[1]{\ensuremath{\vec{#1}}}
\newcommand{\F}[2]{\ensuremath{\frac{#1}{#2}}}
\newcommand{\Q}[1]{\newpage \section*{#1}}
\newcommand{\acc}[1]{\overset{..}{#1}}
\newcommand{\vel}[1]{\overset{.}{#1}}


\pagestyle{fancy}
\lhead{Samuel Loomis}
\setlength{\headheight}{15pt}
\chead{Classical Mechanics HW 7}
\rhead{15 November 2013}
\lfoot{}
\cfoot{\thepage\ of \pageref{LastPage}}
\rfoot{}

\begin{document}

\section*{Question 1, Thornton \& Marion, 7-24}

Consider a simple plane pendulum consisting of a mass $m$ attached to a string of length $l$. After the pendulum is set into motion, the length of the string is shortened at a constant rate. \[\F{\partial{l}}{\partial{t}}=-\alpha=constant\] The suspension point remains fixed. Compute the Lagrangian and Hamiltonian functions. Compare the Hamiltonian and the total energy, and discuss the conservation of energy for the system.
\\
\begin{figure}
\centering
\includegraphics{pendulum.png}
\end{figure}\\
The position of the pendulum at any point in time is:
\[\V{r}=lsin(\theta)\hat{x}-lcos(\theta)\hat{y}\]
The length $l$ and the angle $\theta$ are both functions of time. Taking the time derivative of $r$ gives:
\[\vel{r}=[\vel{l}sin(\theta)+lcos(\theta)\vel{\theta}]\hat{x}-[\vel{l}cos(\theta)-lsin(\theta)\vel{\theta}]\hat{y}\\\]
The velocity will be used in the kinetic energy, $v^2$:
\begin{align*}
\vel{r}^2=&\vel{l}^2sin^2(\theta)+l^2cos^2(\theta)\vel{\theta}^2+2\vel{l}l\vel{\theta}sin(\theta)cos(\theta)\\
&+\vel{l}^2cos^2{\theta}+l^2sin^2(\theta)\vel{\theta}^2-2\vel{l}l\vel{\theta}sin(\theta)cos(\theta)\\
\vel{r}^2=&\vel{l}^2+l^2\vel{\theta}^2
\end{align*}\\
The kinetic energy $T$ is:
\[T=\F{1}{2}m(\vel{l}^2+l^2\vel{\theta}^2)\]
The potential energy of this system is $mgh$, where $h$ is a time dependent function of $l$ and $\theta$:
\[h=-lcos(\theta)\]
The potential energy $U$ is:
\[U=-mglcos(\theta)\]
The Lagrangian becomes:
\[\mathcal{L}=\F{1}{2}m(\vel{l}^2+l^2\vel{\theta}^2)+mglcos(\theta)\]
$\mathcal{H}$
\end{document}


