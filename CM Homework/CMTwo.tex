\documentclass[a4paper,12pt]{article}

\usepackage{fancyhdr}
\usepackage{lastpage}
\usepackage{amsmath}
\usepackage{tikz}
\usepackage{amsfonts}
\pagestyle{fancy}
\lhead{Samuel Loomis}
\setlength{\headheight}{15pt}
\chead{Classical Mechanics HW 2}
\rhead{\thepage\ of \pageref{LastPage}}
\lfoot{}
\cfoot{}
\rfoot{}

\begin{document}

\section*{Question 1}


(a): \[m\frac{dv}{dt}=-mg+c_1v\] 
If $v$ is positive, (moving up), the drag term $c_1v$ would be positive and thus accelerating, this cannot be a drag force.  If $v$ is negative(moving down), the drag term $c_1v$ would also be negative and again, be causing an acceleration, and not a drag, (a) represents neither situation.
\\
(b):\[m\frac{dv}{dt}=-mg-c_1v\]
If $v$ is positive, the drag term would be negative, thus it would be resisting the direction of motion, and thus be a drag force.  If $v$ is negative, the drag term would be positive, thus also resisting the direction of motion. (b) Represents a ball that could be traveling up or down.
\\
(c):\[m\frac{dv}{dt}=-mg+c_2v^2\] 
No matter if the ball is moving up or down, the drag term is positive.  The only time this can represent a drag force is if the ball is moving down, and the positive value would be resisting the motion.  Down only.
\\
(d):\[m\frac{dv}{dt}=-mg-c_2v^2\]
No matter if the ball is moving up or down, the drag term is negative.  The only time this can represent a drag force is if the ball is moving up, and the negative value would be resisting the motion. Up only.
\newpage
\section*{Question 2}
(a): The terminal velocity of an object occurs when the sum of the forces from gravity and drag = 0:
\begin{align*}
0&=mg-c_2v^2\\
mg&=c_2v^2
\end{align*}
Solving for $v$ gives:
\begin{align*}
\frac{mg}{c_2}&=v^2\\
v_{term}&=\sqrt{\frac{mg}{c_2}}
\end{align*}
\\
(b): Terminal velocity for an object subject to both drag forces is a bit more complicated.  The same holds true however, that terminal velocity occurs when the sum of the gravity force and drag forces equals 0:
\begin{align*}
0&=mg-c_1v-c_2v^2
 qw\end{align*}
Defining a few terms to make this cleaner:
\begin{align*}
k_1=\frac{c_1}{m},\ k_2=\frac{c_2}{m}\\
k_2v^2+k_1v-g=0
\end{align*}
Quadratic:
\begin{align*}
a=k_2,\ b=k_1,\ c=g\\
v_{term}=\frac{-k_1\pm\sqrt{(k_1)^2+4k_2g}}{2k_2}
\end{align*}

\newpage
\section*{Question 3}
\begin{align*}
\left|\frac{c_2v^2}{c_1v}\right|=\frac{0.22v\left|v\right|D^2}{(1.55\ast10^{-4})vD}=(1.4\ast10^3)\left|v\right|D
\end{align*}
Ball: $D= 10cm = 0.1m$, $m=200g=0.2kg$\\
(a): I will assume that the quadratic term dominates over the linear term
when the ratio is 20 or higher.  Solving for v:
\begin{align*}
20=(1.4\ast10^3)\left|v\right|D\\
\left|v\right|=\frac{20}{(1.4\ast10^3)D}\approx0.14\frac{m}{s}
\end{align*}
The Quadratic term dominates at velocities above $0.14\frac{m}{s}$.
\\ \\
I will assume the linear term dominates when the inverse ratio is 20 or higher:
\begin{align*}
\frac{1}{20}=(1.4*10^3)|v|D\\
|v|=\frac{1}{20(1.4*10^3)D}\approx 3.6*10^{-4}\frac{m}{s}
\end{align*}
The linear term dominates at speeds below $3.6*10^{-4}\frac{m}{s}$.
\\ \\
(b): The terminal speed occurs when the sum of the forces acting on the ball equals 0:
\begin{align*}
\sum F = 0 &= -mg -c_1v+c_2v^2\\
v_{term}&=\frac{c_1\pm\sqrt{(c_1)^2+4mgc_2}}{2c_2}\\
v_{term}&=\frac{1.55*10^{-4}*.1\pm\sqrt{(1.55*10^{-4}*.1)^2+4*.2*9.8*.22*.1^2}}{2*.22*.1^2}\\
v_{term}&=\frac{1.55*10^{-5}\pm\sqrt{2.4025*10^{-10}+.017248}}{.0044}\\
v_{term}&\approx\frac{1.55*10^{-5}\pm\sqrt{.017248}}{.0044}\\
v_{term}&\approx\frac{1.55*10^{-5}\pm.131332}{.0044}\\
v_{term}&\approx\frac{.13135}{.0044}=29.9\frac{m}{s} (plus)\\
v_{term}&\approx\frac{-.13132}{.0044}=-29.8\frac{m}{s} (minus)\\
\end{align*}
The terminal velocity for the ball is approximately $29.9\frac{m}{s}$.
\\ \\
(c): For the ball, the terminal velocity is in the range of speeds dominated by the quadratic drag term.  If I had to choose one to keep, the quadratic drag would be more accurate, as the linear drag is negligible.

\section*{Question 4}
\begin{align*}
\left|\frac{c_2v^2}{c_1v}\right|=\frac{0.22v\left|v\right|D^2}{(1.55\ast10^{-4})vD}=(1.4\ast10^3)\left|v\right|D
\end{align*}
Oil-Drop: $D= 10^{-4}cm = 10
^{-6}m$, $m=10^{-12}g=10^{-15}kg$\\

(a): Again assuming the quadratic term dominates when the ratio is 20 or higher.  Solving for v:
\begin{align*}
20=(1.4\ast10^3)\left|v\right|D\\
\left|v\right|=\frac{20}{(1.4\ast10^3)D}\approx1.4*10^3\frac{m}{s}
\end{align*}
The Quadratic term dominates at velocities above $1.4*10^3\frac{m}{s}$.
\\ \\
I will assume the linear term dominates when the inverse ratio is 20 or higher:
\begin{align*}
\frac{1}{20}=(1.4*10^3)|v|D\\
|v|=\frac{1}{20(1.4*10^3)D}\approx 3.6*10^{-4}\frac{m}{s}
\end{align*}
The linear term dominates at speeds below $3.6\frac{m}{s}$.
\\ \\
(b): The terminal speed occurs when the sum of the forces acting on the deop\
 equals 0:
\begin{align*}
\sum F = 0 &= -mg -c_1v+c_2v^2\\
v_{term}&=\frac{c_1\pm\sqrt{(c_1)^2+4mgc_2}}{2c_2}\\
v_{term}&=\frac{1.55*10^{-4}*.1\pm\sqrt{(1.55*10^{-4}*10^{-6})^2+4*.2*9.8*.22*.1^2}}{2*.22*.1^2}\\
v_{term}&=\frac{1.55*10^{-5}\pm\sqrt{2.4025*10^{-10}+.017248}}{.0044}\\
v_{term}&\approx\frac{1.55*10^{-5}\pm\sqrt{.017248}}{.0044}\\
v_{term}&\approx\frac{1.55*10^{-5}\pm.131332}{.0044}\\
v_{term}&\approx\frac{.13135}{.0044}=29.9\frac{m}{s} (plus)\\
v_{term}&\approx\frac{-.13132}{.0044}=-29.8\frac{m}{s} (minus)\\
\end{align*}
The terminal velocity for the ball is approximately $29.9\frac{m}{s}$.
\\ \\

\end{document}
