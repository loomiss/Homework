\documentclass[a4paper,12pt]{article}

\usepackage{fancyhdr}
\usepackage{lastpage}
\usepackage{amsmath}
\usepackage{tikz}
\usepackage{amsfonts}
\usepackage{graphicx}

\newcommand{\F}[2]{\ensuremath{\frac{#1}{#2}}}
\newcommand{\V}{\ensuremath{\vec{\boldsymbol{}}}}
\newcommand{\Q}{\newpage \section*{}}
\newcommand{\LP}{\left(}
\newcommand{\RP}{\right)}

\pagestyle{fancy}
\lhead{Samuel Loomis}
\setlength{\headheight}{15pt}
\chead{Classical Mechanics Homework 5}
\rhead{11/04/13}
\lfoot{}
\cfoot{\thepage\ of \pageref{LastPage}}
\rfoot{}

\begin{document}

\section*{Problem 1: Taylor, 2-11}


\Q{Problem 2, Taylor, 2-15}




\Q{Problem 3, Taylor, 2-55}

\Q{Problem 4, Thorton \& Marion, 9-60}

\Q{Problem 5, Taylor 10-6}
Find the CM of a uniform hemispherical shell of inner and outer radii $a$ and $b$ and mass $M$.

(a) Comment on the limiting case when $a\rightarrow 0$.

(b) Comment on the limiting case when $a\rightarrow b$.\\ \\
The integral equation for the position of the center of mass is:
\[\V{R}=\F{1}{M}\int\V{r}dm\]
$dm$ stands for a bit of mass.  The bit of mass can be described as the volume mass density, $\rho(\V{r})$, times a bit of volume, $dV$. $\rho(\V{r})$ is uniform and thus will be represented by a simple $\rho$.\\ \\
A hemisphere is most easily handled when dealing with spherical coordinates.\\
$dV$ in spherical is represented by:
\[dV=r^2sin\theta drd\phi d\theta\]
$\V{r}$ in spherical is simply $r\hat{r}$, however, over the $\phi$ and $\theta$ integrals, the direction of $\V{r}$ changes, and thus $\V{r}$ needs to be taken to a coordinate system that doesn't change over the integrals.
$\V{r}$ expressed in cartesian is:
\[\V{r}=rsin\theta cos\phi \hat{x}+rsin\theta sin\phi\hat{y}+rcos\theta\hat{z}\]
$\V{R}$ can now be expressed as a spherical volume integral:
\begin{align*}
\V{R}=\F{1}{M}\int\V{r}dm&=\F{1}{M}\int\rho(rsin\theta cos\phi \hat{x}+rsin\theta sin\phi\hat{y}+rcos\theta\hat{z})r^2sin\theta drd\phi d\theta
\end{align*}
\end{document}


