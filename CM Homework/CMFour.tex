\documentclass[a4paper,12pt]{article}

\usepackage{fancyhdr}
\usepackage{lastpage}
\usepackage{amsmath}
\usepackage{tikz}
\usepackage{amsfonts}
\pagestyle{fancy}
\lhead{Samuel Loomis}
\setlength{\headheight}{15pt}
\chead{Classical Mechanics HW 4}
\rhead{\thepage\ of \pageref{LastPage}}
\lfoot{}
\cfoot{}
\rfoot{}

\begin{document}

\subsection*{1. Taylor, Problem 3.10}
Consider a rocket(initial mass $m_0$ accelerating from rest in free space).  At first, as it speeds up, its momentum $p$ increases, but as its mass $m$ decreases $p$ eventually begins to decrease.  For what value of $m$ is $p$ maximum?\\
\\
The total momentum of the rocket at time t is:
\begin{align*}
P(t)=mv
\end{align*}
A short time later $t+dt$ the mass of the rocket has changed to $m+dm$ and the velocity of the rocket has changed to $v+dv$.  There has also been fuel ejected, the mass of the fuels is $-dm$ and it's velocity compared to a stationary frame is now $v-v_{ex}$. The total momentum of the rocket and the ejected fuel at this new time, ignoring the double infitessimal $dmdv$ is:
\begin{align*}
P(t+dt)=(m+dm)(v+dv)-dm(v-v_{ex})=mv + mdv +v_{ex}dm
\end{align*}
The change in total momentum over this time period is:
\begin{align*}
dP=P(t+dt)-P(t)=mdv+v_{ex}dm
\end{align*}
The change in total momentum is 0, because no external forces are acting apon the system. Thus:
\begin{align*}
mdv=-v_{ex}dm
\end{align*}
Separation of variables allows us to solve for the velocity of the rocket as a function of its mass:
\begin{align*}
dv=-v_{ex}\frac{dm}{m}
\end{align*}
If the exhaust velocity is constant, integration gives:
\begin{align*}
v-v_0=v_{ex}ln(m_0/m)
\end{align*}









\subsection*{2. Taylor, Problem 3.14}
Consider a rocket subject to a linear resistive force,
$f = −bv$, but no other external forces. Use the equation $m\dot{v} = −v_{ex}\dot{m} +
F_{ext}$ (from Problem 3.11) to show that if the rocket starts from rest and
ejects mass at a constant rate $ k = −\dot{m}$ , then its speed is given by
\begin{align*}
v = \frac{k}{b}v_{ex}\left[1-\left(\frac{m}{m_0}\right)^{b/k}\right]
\end{align*}
\end{document}


